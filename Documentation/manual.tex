
\documentstyle[12pt,thmsa,sw20lart]{article}
%%%%%%%%%%%%%%%%%%%%%%%%%%%%%%%%%%%%%%%%%%%%%%%%%%%%%%%%%%%%%%%%%%%%%%%%%%%%%%%%%%%%%%%%%%%%%%%%%%%%%%%%%%%%%%%%%%%%%%%%%%%%
%TCIDATA{TCIstyle=article/art4.lat,lart,article}

%TCIDATA{Created=Sun Feb 13 20:08:19 2005}
%TCIDATA{LastRevised=Mon Feb 14 00:46:15 2005}

\input{tcilatex}
\begin{document}


\section{Introduction}

FermiQCD is a software library for fast development of parallel Lattice QCD
code. It includes examples and applications. The latest version of FermiQCD
supports:

\begin{itemize}
\item  Fully Object Oriented design

\item  Natural syntax

\item  Supports Wilson, Staggered and Domain-wall fermions

\item  Parallelization based on MPI is hidden to high level programmer

\item  SU3 operations optimized for P4 (SSE and SSE2 instructions)

\item  PSIM technology for emultaing parallel processes on a single
processor machine and improving performance on multi-threaded processors

\item  Compile and runs on multiple architectures including Linux, Mac and
Windows (with cygwin).
\end{itemize}

\subsection{Installation}

\begin{enumerate}
\item  Download the file fermiqcd\_4.0.tar.gz

\item  Execute: gunzip fermiqcd\_4.0.tar.gz

\item  Execute: tar xvf fermiqcd\_4.0.tar.gz
\end{enumerate}

\subsection{Files}

Upon installation fermiqcd creates the following directory structure
\begin{verbatim}
\FermiQCD
\FermiQCD\Version_4.0
\FermiQCD\Version_4.0\Libraries
\FermiQCD\Version_4.0\Documentation
\FermiQCD\Version_4.0\Examples
\FermiQCD\Version_4.0\Converters
\FermiQCD\Version_4.0\Tests
\FermiQCD\Version_4.0\Other
\end{verbatim}

The folder ``Libraries'' contains the software libraries. There is no need
to precompile anything since all the code is in the header files. This is
done to simplify usage and allow the compiler to perform better template
optimizations.

The files starting with {\tt mdp\_} belong to the Matrix Distributed
Processing (MDP) library, required by FermiQCD. The files starting with {\tt %
fermiqcd\_} are the proper FermiQCD files. They are distrubuted together but
are covered by different licenses. 

The folder ``Documentation'' contains licenses and documentation for MDP and
FermiQCD. MDP and FermiQCD cannot be redistributed without this folder.

The folder ``Examples'' contains all of the examples described in these
tutorial.

The folder ``Converters'' contains converters for starndard QCD file formats
into the MDP file format used by FermiQCD. 

You may ask: why another file format? Technically FermiQCD itself does not
specify a file format but it defines fields of objects defined on a lattice.
Field objects inherit parallel load/save operations from the underlying MDP
library which specifies a single file format for any generic field (any
number of lattice dimensions, any structure at the site) optimized for
parallel IO. None of the previous file formats was general enough since they
are all specific for a type of field and usually for 4 dimensions.

In any case FermiQCD can read UKQCD, MILC, CANOPY and many ASCII file
formats. If you format is not supported email me and I will send you a
converted in a week at no charge.

The folder ``Tests'' contains programs and libraries that I consider a work
in progress. They are included with the official distribution to allow
people to contribute.

The folder ``Other'' contains examples of applications in fields other than
QCD, for example Cellular Authomata.

\subsection{Compilation instructions}

To compile all the example go into /FermiQCD/Version\_4.0/Examples and type
\begin{verbatim}
make all
\end{verbatim}

or compile any of the programs with
\begin{verbatim}
g++ prg.cpp -I../Libraries -o prg.exe -O3 [options]
\end{verbatim}

You may want to edit the first few lines of the Makefile in order to enable
some specific compiler options:
\begin{verbatim}
g++ prg.cpp -I../Libraries -o prg.exe --O3 -DLINUX
\end{verbatim}

will compile for Linux and be able to measure running time
\begin{verbatim}
g++ prg.cpp -I../Libraries -o prg.exe -O3 -DUSE_DOUBLE_PRECISION
\end{verbatim}

will compile the algorithms to use double precision
\begin{verbatim}
g++ prg.cpp -I../Libraries -o prg.exe -O3 -DSSE2
\end{verbatim}

will compile the algorithms to use SSE/SSE2 optimizations (only on P4
processors)
\begin{verbatim}
g++ prg.cpp -I../Libraries -o prg.exe -O3 -DPARALLEL
\end{verbatim}

will compile with MPI, otherwise it will compile with PSIM Parallel
SIMulator (enables emulating parallel processing on a single processor PC,
it is recommended on a multi-threaded processor where it will increase
speed).

Compiler options can be combined.

\subsection{A first program}

\subsection{General principles}

More or less any FermiQCD the same structure:
\begin{verbatim}
#include ''fermiqcd.h''
int main(int argc, char** argv) {
   mdp.open_wormholes(argc,argv);
   // declare conventions
   // declare lattices
   // declare fields
   // declare variables
   // run algorithms
   mdp.close_wormholes();
   return 0;
}
\end{verbatim}

Where open\_wormholes and close\_wormholes respectively start and stop
communications.

Conventions are declared using the command
\begin{verbatim}
declare_base_matrices(''FERMIQCD'');
\end{verbatim}

which basically declares gamma matrices (Gamma[], Gamma5, and Gamma1). Other
conventions supported include ``UKQCD''.

A lattice, for examle a 4D lattice $16\times 8^3$ called {\tt mylattice} can
be declared as follows
\begin{verbatim}
int box[]={16,8,8,8};
mdp_lattice lattice(4,box);
\end{verbatim}

The constructor of the class can take optional parameters that are discussed
in the next section. The optional parameters specify how the lattice object
has to be partionioned over the parallel processes (default: by timeslices),
specify the lattice topology (default: mesh) and the size of the buffer used
for parallel communications (default: optimized for Wilson and Clover
actuions).

A {\bf lattice object} contains a parallel random number generator. To ask
every site to print a uniform random number:
\begin{verbatim}
#include ''fermiqcd.h''
int main(int argc, char** argv) {
   mdp.open_wormholes(argc,argv);
   declare_base_matrices(''FERMIQCD'');
   int box[]={16,8,8,8}
   mdp_lattice mylattice(4,box);
   site x(mylattice);
   forallsites(x) 
      cout<<mylattice.random(x).plain()<<endl;
   mdp.close_wormholes();
   return 0;
}
\end{verbatim}

Class {\bf site} represents a site of the lattice; {\tt forallsites(x)} is a
parallel loop over all lattice sites using the site {\tt x} as looping
variable and 
\begin{verbatim}
mylattice.random(x)
\end{verbatim}

is the {\bf random number generator} associated to the site {\tt x}. {\tt %
plain()} is a method of the random generator that returns a uniform random
number in (0,1). Other methods include
\begin{verbatim}
mylattice.random(x).SU(n)
\end{verbatim}

that generates a random SU(n) matrix using Cabibbo-Marinari.

To declare a {\bf field} of floating point numbers in single precision
called {\tt myfield} on our lattice
\begin{verbatim}
mdp_field<float> myfield(mylattice);
\end{verbatim}

To {\bf save} or {\bf load} a field\footnote{%
This functions may crash if you there is not enough memory to allocate
buffers for parallel IO. If this occurs pass a second {\tt int} argument to
the load/save functions with a value below 1024. The smoller the value, the
smaller the required buffer size.}
\begin{verbatim}
string filename;
myfield.save(filename);
myfield.load(filename);
\end{verbatim}

The following code creates a filed of floating point numbers, initialize
them at random and saves them
\begin{verbatim}
#include ''fermiqcd.h''
int main(int argc, char** argv) {
   mdp.open_wormholes(argc,argv);
   declare_base_matrices(''FERMIQCD'');
   int box[]={16,8,8,8}
   mdp_lattice mylattice(4,box);
   mdp_field<float> myfield(mylattice);
   site x(mylattice);
   forallsites(x) 
      myfield(x)=mylattice.random(x).plain();
   myfield.update();
   myfield.save(''myfield.mdp'');
   mdp.close_wormholes();
   return 0;
}
\end{verbatim}

Note the function {\bf update()}. It is the single and most important
function in MDP and FermiQCD. It must be called every time a field changes
and before it is used. It instruct the parallel nodes to keep the copies of
the lattice sites and field variables therein sinchronized.

\subsection{Notes}

To loop over all sites of a given parity (EVEN or ODD)
\begin{verbatim}
forallsitesofparity(x,EVEN)
\end{verbatim}

To loop over all local sites and local copies of sites stored on other
processors
\begin{verbatim}
forallsitesandcopies(x)
\end{verbatim}

(This is used when looping to initialize a field with a local expression
that does not require the parallel random number generator, in order to
avoid a susequent call to the function update; if not sure, do not use it.)

To loop over all local even sites and local copies of even sites stored on
other processors:
\begin{verbatim}
forallsitesandcopiesofparity(x,EVEN)
\end{verbatim}

(Same as above.)

To print the time components of a site variable x
\begin{verbatim}
for(int k=0;k<x.lattice().ndim;k++)
   cout << x[k] << endl;
\end{verbatim}

Where {\tt x.lattice().ndim} reads as the number of dimensions ({\tt ndim})
of the lattice on which the site {\tt x} was declared. All fields and site
objects have a method lattice() to obtain a reference to the underlying
lattice. {\tt x[k]} reads as the {\tt k}-th coordinate of the site {\tt x}.
If the lattice was 4D x has four components 0,1,2, and 3.\ We adopt the
convention of calling coordinate 0 the TIME\ coordinate and 1,2,3 the space
coordinates.

\subsection{Other fields}

FermiQCD comes with a set of predefined fields
\begin{verbatim}
gauge_field
fermi_field
fermi_propagator
staggered_field 
staggered_propagator
dw_fermi_field
dw_fermi_propagator
\end{verbatim}

and algorithms to create and used them. All the FermiQCD algorithms work
with any SU(n) gauge group although they are highly optimized for SU(3). The
staggered algorithms also work for any even-dimensional space. Some other
algorithms require a four dimensional space because of optimizations related
to the gamma matrix conventions.

\section{Matrix Distributed Processing}

\section{Quantum Chromo Dynamics}

\subsection{Pure gauge}

A gauge field is defined as 
\[
U_\mu (x)=e^{iaA_\mu (x+\frac a2\widehat{\mu })}\simeq 1+iaA_\mu (x+\frac a2%
\widehat{\mu })
\]
where $a$ is the lattice spacing and $A_\mu (x)$ is gauge field in $SU(n)$
and $\widehat{\mu }$ is a unit versor in direction $\mu $.

\subsubsection{Creating a hot gauge configuration}
\begin{verbatim}
#include ''fermiqcd.h''
int main(int argc, char** argv) {
   mdp.open_wormholes(argc,argv);
   declare_base_matrices(''FERMIQCD'');
   int box[]={16,8,8,8}
   mdp_lattice mylattice(4,box);
   int nc=3;
   gauge_field U(mylattice,nc);
   set_hot(U);
   U.save(''gauge.0000.mdp'');
   mdp.close_wormholes();
   return 0;
}
\end{verbatim}

the function set\_hot is already implemented but could have been implemented
easily as 
\begin{verbatim}
void set_hot(gauge_field &U) {
   site x(U.lattice());
   forallsites(x)
      for(int mu=0; mu<U.ndim; mu++)
         U(x,mu)=U.lattice().random(x).SU(U.nc);
   U.update();
}
\end{verbatim}

\subsubsection{Creating a cold gauge configuration}
\begin{verbatim}
#include ''fermiqcd.h''
int main(int argc, char** argv) {
   mdp.open_wormholes(argc,argv);
   declare_base_matrices(''FERMIQCD'');
   int box[]={16,8,8,8}
   mdp_lattice mylattice(4,box);
   int nc=3;
   gauge_field U(mylattice,nc);
   set_cold(U);
   U.save(''gauge.0000.mdp'');
   mdp.close_wormholes();
   return 0;
}
\end{verbatim}

the function set\_hot is already implemented but could have been implemented
easily as 
\begin{verbatim}
void set_cold(gauge_field &U) {
   site x(U.lattice());
   forallsites(x)
      for(int mu=0; mu<U.ndim; mu++)
         U(x,mu)=1;
   U.update();
} 
\end{verbatim}

Note how each site initializes {\tt U(x,mu)} with 1 (interpreted as $3\times
3$ identity matrix). Since every site variable is initialized with a
constant and it does not depend on the random number generator it would be
more efficient to ask each process to initialize also the local copies of
remote site variables and avoid calling update
\begin{verbatim}
void set_cold(gauge_field &U) {
   site x(U.lattice());
   forallsitesandcopies(x)
      for(int mu=0; mu<U.ndim; mu++)
         U(x,mu)=1;
} 
\end{verbatim}

This is how set\_cold is implemented in practice and it requires no parallel
communication.

\subsubsection{Performing Wilson heatbath steps}

The basic ingredient of any Lattice computation is the Markov Chain Monte
Carlo that creates different gauge field configurations $U_\mu ^{[k]}(x)$
randomly distributed with probability 
\[
P[U]=e^{-S_E[U]}
\]
where $S_E$ is the Euclidean gauge action expressed in terms of $U$.

The $U_\mu ^{[k]}(x)$ are typically generated using an iterative procedure
\[
U_\mu ^{[k]}(x)\rightarrow U_\mu ^{[k+1]}(x)
\]

where $U_\mu ^{[0]}(x)$ is set either hot or cold and the algorithm to go
from one to the next implements the action $S_E$. The most common algorithm
is the heatbath and the simplest discretization of the gauge action is given
by
\begin{eqnarray*}
S_E[U] &=&\beta \sum_{x,\mu ,\nu }[U_\mu (x)U_\nu (x+\widehat{\mu })U_\mu
^H(x+\nu )U_\nu ^H(x)-1] \\
&\simeq &\frac{a^4}{4g^2(a)}\sum_xG^{\mu \nu }(x)G_{\mu \nu }(x)
\end{eqnarray*}
where $\beta =g^{-2}(a)$ is the regularized coupling constants used as input
to set the lattice scale $a$ and $G^{\mu \nu }(x)=\partial _\mu A_\nu
-\partial _\nu A_\mu +[A_\mu ,A_\nu ]$ is the chromo-electro-magnetic field.

Here is the code to generate 100 gauge configurations $U_\mu ^{[k]}(x)$ and
save them.
\begin{verbatim}
#include ''fermiqcd.h''
int main(int argc, char** argv) {
   mdp.open_wormholes(argc,argv);
   declare_base_matrices(''FERMIQCD'');
   int box[]={16,8,8,8}
   mdp_lattice mylattice(4,box);
   int nc=3;
   gauge_field U(mylattice,nc);
   set_cold(U);
   coefficients gauge;
   gauge[''beta'']=6.0;
   int niter=10;
   for(int k=0; k<100; k++) {
      WilsonGaugeAction::heatbath(U,gauge,niter);
      U.save(string(''gauge.'')+tostring(k)+string(''.mdp''));
   }
   mdp.close_wormholes();
   return 0;
}
\end{verbatim}

Note that gauge is a variable of type {\bf coefficients}, basically a hash
table that associates a floating point to any string. Variables of type
coefficients are used to pass parameters (or coefficients) to algorithms
that implement a physical action.

{\tt WilsonGaugeAction::heatbath} is the {\bf hetbath} algorithm using the
Wilson Gauge Action (who would have guessed?). It's first argument is the
gauge field is acts upon (reads and writes it), the second argument is the
set of coefficients (the only one it needs it ''beta'', i.e. the lattice
spacing), the third argument is the number of iterations before returning.
Noe that the number of interations technically is not a coefficient of the
action therefore it not passed as a variable in the gauge object.

The line 
\begin{verbatim}
string(''gauge.'')+tostring(k)+string(''.mdp'')
\end{verbatim}

simply builds a filename without messing around with pointers since they are
unsafe.

\subsubsection{Improved actions}

All actions in FermiQCD are implemented as pure static classes (i.e. classes
with no member variables and only static methods). All gauge actions have a
static method heatbath that mush follow the same prototype.

For the Wilson gauge action
\begin{verbatim}
coefficients gauge;
gauge[''beta'']=6.0;      
WilsonGaugeAction::heatbath(U,gauge,niter);
\end{verbatim}

For the MILC improved gauge action
\begin{verbatim}
coefficients gauge;
gauge[''beta'']=6.0;      
gauge[''zeta'']=1.0;      
gauge[''u_t'']=1.0;      
gauge[''u_s'']=1.0;      
string model=''MILC'';
ImprovedGaugeAction::heatbath(U,gauge,niter,model);
\end{verbatim}

For the Morningstar improved gauge action
\begin{verbatim}
coefficients gauge;
gauge[''beta'']=6.0;      
gauge[''zeta'']=1.0;      
gauge[''u_t'']=1.0;      
gauge[''u_s'']=1.0;      
string model=''Morningstar'';
ImprovedGaugeAction::heatbath(U,gauge,niter,model); 
\end{verbatim}

\subsubsection{Average plaquette}

A plaquette is defined as
\begin{eqnarray*}
P_{\mu \nu }(x) &=&U_\mu (x)U_\nu (x+\widehat{\mu })U_\mu ^H(x+\nu )U_\nu
^H(x) \\
&\simeq &a^2G_{\mu \nu }
\end{eqnarray*}
here is the code to compute
\[
\frac 16\frac 1{N_V}\frac 1{N_c}\func{Re}Tr\sum_{x,\mu >\nu }P_{\mu \nu }(x)
\]
where $N_V$ is the lattice volume and $N_c$ is the number of colors.
\begin{verbatim}
#include ''fermiqcd.h''
int main(int argc, char** argv) {
   mdp.open_wormholes(argc,argv);
   declare_base_matrices(''FERMIQCD'');
   int box[]={16,8,8,8}
   mdp_lattice mylattice(4,box);
   int nc=3;
   gauge_field U(mylattice,nc);
   set_cold(U);
   coefficients gauge;
   gauge[''beta'']=6.0;
   int niter=10;
   for(int k=0; k<100; k++) {
      WilsonGaugeAction::heatbath(U,gauge,niter);
      U.save(string(''gauge.'')+tostring(k)+string(''.mdp''));
      mdp << average_plaquette(U) << endl;
   }
   mdp.close_wormholes();
   return 0;
}
\end{verbatim}

The function average\_plaquette computes the average plaquette on the gauge
field U. The output is a float number. Sending the output to mdp rather than
cout makes sure only process 0 prints the average plaquette even if all
processes contribute to the computation.

There is a similar function
\begin{verbatim}
average_plaquette(U,mu,nu)
\end{verbatim}

that computes the average plaquette considering the mu-nu plane only where
mu and nu are integers
\[
\frac 1{N_V}\frac 1{N_c}\func{Re}Tr\sum_xP_{\mu \nu }(x)
\]

The function average plaquette is already implemented as
\begin{verbatim}
myreal average_plaquette(gauge_field &U,int mu,int nu) {
 myreal tmp=0;
 site x(U.lattice());
 forallsites(x)
   tmp+=real(trace(plaquette(U,x,mu,nu)));
 mdp.add(tmp);
 return tmp/(U.lattice().nvol_gl*U.nc);
}
\end{verbatim}

where {\tt plaquette(U,x,mu,nu)} is defined as
\begin{verbatim}
U(x,mu)*U(x+mu,nu)*hermitian(U(x,nu)*U(x+nu,mu)); 
\end{verbatim}

mdp.add(tmp) adds the values tmp computed by the parallel processes (always
to be called after summing a variable inside a forallsites loop), and {\tt %
U.lattice().nvol\_gl} reads as the total number of lattice sites ({\tt %
nvol\_gl}) in the lattice for {\tt U}.

\subsubsection{Average path}

Note how the most general gauge observable has the form
\[
\oint_Ce^{iA_\mu dx^\mu }
\]
where $C$ is a generic path. Here we want to compute

\[
\frac 1{N_V}\frac 1{N_c}\func{Re}Tr\sum_x\oint_Ce^{iA_\mu dx^\mu }
\]
and here is the code:
\begin{verbatim}
#include ''fermiqcd.h''
int main(int argc, char** argv) {
   mdp.open_wormholes(argc,argv);
   declare_base_matrices(''FERMIQCD'');
   int box[]={16,8,8,8}
   mdp_lattice mylattice(4,box);
   int nc=3;
   gauge_field U(mylattice,nc);
   set_cold(U);
   coefficients gauge;
   gauge[''beta'']=6.0;
   int niter=10;
   int mu=0, nu=1;
   int path[6][2]={{+1,mu},{+1,mu},{+1,nu},{-1,mu},{-1,nu},{-1,nu}};
   for(int k=0; k<100; k++) {
      WilsonGaugeAction::heatbath(U,gauge,niter);
      U.save(string(''gauge.'')+tostring(k)+string(''.mdp''));
      mdp << average_path(U,6,path) << endl;
   }
   mdp.close_wormholes();
   return 0;
}
\end{verbatim}

The function {\tt average\_path} is a computes the average path on the gauge
field U. where the path $C$ is a spacified by a 2D array {\tt d} of links
and each each link is a verse (+1 or -1) and a direction (mu,
nu=0,1,2,3,...). Note that because a path may be highly non-local the
implementation of the funciton average\_path requires, in general, quite
some communication.

This function is already implemented and looks like the following: 
\begin{verbatim}
mdp_complex average_path(gauge_field &U, int length, int d[][2]) {
  mdp_matrix_field psi1(U.lattice(),U.nc,U.nc);
  mdp_matrix_field psi2(U.lattice(),U.nc,U.nc);
  mdp_site x(U.lattice());
  mdp_complex sum=0;
  for(int i=0; i<length; i++) {
    if(i==0) 
       forallsites(x) 
          psi1(x)=U(x,d[i][0],d[i][1]);
    else
       forallsites(x) 
          psi1(x)=psi1(x)*U(x,d[i][1],d[i][1]);
    if(i<length-1) {
       psi1.update();    
       if(d[i][0]==+1)
          forallsites(x) psi2(x)=psi1(x+d[i][1]);
       else if(d[i][0]==-1)
          forallsites(x) psi2(x)=psi1(x-d[i][1]);
       psi1=psi2;
    }
  }
  forallsites(x) sum+=trace(psi1(x));
  return sum/(U.lattice().nvol_gl*U.nc);
}
\end{verbatim}

\subsubsection{Chromo-electro-magnetic field}

The Chromo-electro-magnetic field $P_{\mu \nu }=a^2G_{\mu \nu }$ has its own
class as a field of vectors of matrices but one never really needs to
declare it since it is uniquely associated to a aguge field. Given a gauge
field {\tt U} just call:
\begin{verbatim}
compute_em_field(U);
\end{verbatim}

and to obtain the plaquette $P_{\mu \nu }(x)$ just call
\begin{verbatim}
U.em(x,mu,nu)
\end{verbatim}

Therefore the average plaquette could also have been computes as
\begin{verbatim}
mdp_real sum=0.0;
compute_em_field(U);
forallsites(x) 
   for(int mu=0; mu<U.ndim; mu++)
      for(int nu=mu+1; nu<U.ndim; nu++)
         sum+=real(trace(U.em(x,mu,nu));
retur sum/(U.lattice().nvol_gl*U.ndim*(U.ndim-1)/2*U.nc);
\end{verbatim}

\subsection{Fermions and inverters}

\subsection{Wilson fermions}

\subsection{Wilson mesons}

\subsection{Staggered fermions}

\subsection{Staggered mesons}

\subsection{Domain-wall fermions}

\subsection{Domain-wall mesons}

\subsection{Gauge fixing}

\subsection{Smearing}

\subsection{All-to-all propagators}

\subsection{Converting file formats}

\subsection{Implementing a new type of field}

\subsection{Implementing a new actions}

\end{document}
